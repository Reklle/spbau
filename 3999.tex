\documentclass{article}
\usepackage[utf8]{inputenc}
\usepackage{amsmath}
\usepackage{mathtools}
\usepackage{physics}
\usepackage{xcolor}
\usepackage{ulem}
\usepackage{amssymb}
\usepackage[russian]{babel}

\newcommand{\why}{\sout}
\newcommand{\shouldwestopanonymity}{\thispagestyle{empty}}
\def  {empty}

\begin{document}
\shouldwestopanonymity

\subsection*{The confidentiality war is lost, so what?}
I have no secret information, so I can use the Internet freely, ignoring issues of privacy and other. At present it is almost impossible to be online without an ad blocker. Fortunately, there are plenty of them.\\
But how do you keep your privacy when you need that? Without radical measures, the risks of de-anonymization are large enough in modern world! But these measures are really terrible. You need to have a computer with no Internet connection and it is advisable not to connect devices that have had any contact with the Internet. This is the only way to ensure maximum data security. But you always have to deal with something online that’s going to be dangerous with some chances. For example, a computer that is not connected to the Web can be used to check files against the virus before booting for greater security.\\
As practice shows, it is impossible to avoid de-anonymization with half measures. Struggle is impossible. \why{Why not just give up?}\\
Futhermore, even big size of files is not an actual problem. There is turbo mode in browsers and other stuff. \why{Giving up doesn’t mean losing?}\\
\subsection*{Retesting of the experiment results and comments}

I didn’t find 100 cookies on the analyzed site. This indicates the irrelevance of the study. However, this does not indicate the falsity of the study, \why{but it says we should give up and stop trying to protect our information} .
Indeed, the site now weighs about 30 Mb. However, dataskydd.net counted only 6 cookies. This raises suspicions about the experiment.\\
In addition, the author’s own article takes about 20MB. This means that 95\% of the data transferred between you and the has nothing to do with the article. \why{Why do we need anonymity?} Tips for improving network security are really good \why{but we have nothing to hide}. Ad block is the most important plug-in for browser. I would suggest \why{stop anonymity} "Enhancer for YouTube" and "I don't care about cookies". I have been using them for a long time. \why{Why?}

\vspace{90}
\textbf{Item #:} $\blacksquare\blacksquare\blacksquare-\blacksquare\blacksquare\blacksquare\blacksquare$\\
\textbf{Object Class:} Apollyon\\
Special Containment Procedures: cannot be contained at the present moment\\
\why{stop}\\
\why{anonymity}\\
\why{preserve some remnants}\\
...

\end{document}